\chapter{Conclusão}\label{cap:conclusoes}

Com o objetivo de criar um modelo de classificação para classificar o estado de alagamento das ruas da cidade do Rio de Janeiro,
esta pesquisa organizou um conjunto de dados original ao longo de meses de captação e classificação de imagens do sistema de câmeras do \acrshort{cor}, denominado \acrshort{rfd}, disponibilizado em \href{https://doi.org/10.5281/zenodo.15670835}{10.5281/zenodo.15670835}.
Com este conjunto de dados, foi realizada uma comparação entre as arquiteturas \acrshort{vgg}, \textit{Inception}, \textit{DenseNet}, \textit{MobileNet} e \Acrshort{vit} no problema de detecção de alagamentos,
para achar o melhor modelo entre as arquiteturas selecionadas para alcançar o objetivo do trabalho.

Os resultados mostraram que o \acrshort{vit} superou as outras arquiteturas com diferentes taxas de aprendizagem.
A alteração da intensidade luminosa da imagem teve impacto claro no desempenho do modelo e ajudou a diminuir a diferença de acurácia entre as imagens noturnas e diurnas,
em comparação aos resultados descritos em Piedad \textit{et al.} \cite{piedad2022}.

O modelo \acrshort{vit} se destacou usando \acrshort{sgd} como otimizador e com taxa de aprendizagem de 1e$^{-4}$.
Foi o melhor não só pelas suas métricas, mas pela sua maior estabilidade na curva de perda e pela menor discrepância entre acurácia de treino e validação.

Todos os modelos foram treinados utilizando imagens capturadas pelo sistema de câmeras do \acrshort{cor}, onde o \acrshort{vit} obteve acurácia de 83\%.
Este trabalho também analisou o modelo obtido em outras fontes de dados; o modelo conseguiu acurácia de 73\% no conjunto de dados \acrshort{efd} \cite{BarzSchroeterMuench2018_1000117723},
e de 74\% no conjunto de dados de Sazara \textit{et al.} \cite{sazara2019}.

Ao abordar o complexo problema de classificar imagens de alagamento de uma cidade grande e diversa como o Rio de Janeiro,
este trabalho conseguiu alcançar seu objetivo de concluir uma abordagem inicial para a classificação automática da situação de alagamento de ruas,
através de um conjunto de dados organizado e disponibilizado por esta dissertação, baseado na própria cidade do Rio de Janeiro (RJ).

\section{Trabalhos futuros}

Nesta dissertação, foi utilizado em todo o conjunto de treino o fator de claridade para diminuir o efeito de claridade excessiva em algumas imagens.
Buscar uma abordagem que afete somente as imagens que possuam esse problema, e até mesmo somente na região com tal claridade.

A integração de diferentes conjuntos de dados ao treinamento do modelo pode impactar positivamente seu desempenho.
Embora diferentes quantidades de tráfego tenham sido utilizadas no treinamento, outros eventos que podem ocorrer na rua ou ao seu redor, como manutenção das vias ou construções,
não foram incluídos no conjunto de dados de treinamento.
Portanto, essas situações poderiam ser classificadas incorretamente pelo modelo, e imagens desses eventos deveriam ser incluídas no conjunto de dados de treinamento.

Apesar do otimizador \acrshort{sgd} inicialmente escolhido apresentar melhor desempenho,
os resultados com o \acrshort{nadam} mostram que mais estudos sobre a escolha de diferentes valores para os parâmetros do otimizador podem aprimorar o modelo atual.

Além disso, alterações na arquitetura, como a criação de novas camadas demonstrada em Agung \textit{et al.} \cite{agung2023},
podem melhorar o desempenho do modelo, uma vez que eles alcançaram uma precisão de 0,95 para o \textit{MobileNetV2},
enquanto este trabalho alcançou uma precisão mais baixa de 0,78 com a mesma arquitetura.

O uso de outros instrumentos meteorológicos de medição, como pluviômetros instalados por toda a cidade, pode ser usado para corroborar as classificações resultantes do modelo gerado.
Modelos baseados em imagens de satélites seriam importantes para uma maior eficácia de previsões.
Outros aspectos importantes a serem incluídos são dados ligados ao histórico de regiões de alagamentos e à própria geografia da cidade, onde lugares próximos de encostas e mais baixos são as regiões que sabidamente devem ser mais monitoradas.
No entanto, como o escopo da pesquisa foi o desenvolvimento de um modelo de classificação e algumas ruas não possuem sensores de nível de água,
esse tipo de análise foi deixado de lado para uma abordagem inicial.
