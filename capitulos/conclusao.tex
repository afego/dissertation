\chapter{Conclusão}\label{cap:conclusoes}

Com o objetivo de criar um modelo de classificação para classificar o estado de alagamento das ruas da cidade do Rio de Janeiro,
esta pesquisa criou um conjunto de dados original de 4620 através de meses de captação e classificação de imagens da cidade do Rio, que está disponibilizado \href{https://github.com/afego/computervision}{neste repositório de GitHub}.
Com este conjunto de dados, foi realizada uma comparação entre as arquiteturas \acrshort{vgg}, \textit{Inception}, \textit{DenseNet}, \textit{MobileNet} e \Acrshort{vit} no problema de detecção de alagamentos
para achar o melhor modelo entre arquiteturas selecionadas para alcançar o objetivo do trabalho.

Os resultados mostraram que o \acrshort{vit} geralmente superou as outras arquiteturas com diferentes taxas de aprendizagem. 
A alteração da intensidade luminosa da imagem teve impacto claro no desempenho do modelo e ajudou a diminuir a diferença da acurácia entre as imagens noturnas e diurnas, 
em comparação aos resultados descritos em Piedad \textit{et al.} \cite{piedad2022}.

O modelo que se destacou, usando \acrshort{sgd} como otimizador e com taxa de aprendizagem de 1e$^{-4}$ foi o melhor não só pelas suas métricas 
mas pela sua maior estabilidade na curva de perda e pela menor discrepância entre acurácia de treino e validação.

Todos os modelos foram treinados utilizando imagens capturadas pelo sistema de câmeras COR, onde o \acrshort{vit} obteve acurácia de 83\%. 
Este trabalho também analisou o modelo obtido em outras fontes de dados, o modelo conseguiu acurácia de 73\% no conjunto de dados \acrshort{efd} \cite{BarzSchroeterMuench2018_1000117723}, 
e de 74\% no conjunto de dados de Sazara \textit{et al.} \cite{sazara2019}.

Ao abordar o complexo problema de classificar imagens de alagamento de uma cidade grande e diversa como o Rio de Janeiro, 
este trabalho conseguiu alcançar seu objetivo de concluir uma abordagem inicial para a classificação automática da situação de alagamento de ruas, 
através de um conjunto de dados original baseado na própria cidade do Rio.

\section{Trabalhos futuros e considerações}

A integração de diferentes conjuntos de dados ao treinamento do modelo pode impactar positivamente seu desempenho.
Embora diferentes quantidades de tráfego tenham sido utilizadas no treinamento, outros eventos que podem ocorrer na rua ou ao seu redor, como manutenção das vias ou construções, 
não foram incluídos no conjunto de dados de treinamento. 
Portanto, essas situações poderiam ser classificadas incorretamente pelo modelo, e imagens desses eventos deveriam ser incluídas no conjunto de dados de treinamento.

Apesar do otimizador \acrshort{sgd} inicialmente escolhido apresentar melhor desempenho, 
os resultados com o \acrshort{nadam} mostram que mais estudos sobre a escolha de diferentes valores para os parâmetros do otimizador podem aprimorar o modelo atual.

Além disso, alterações na arquitetura como a criação de novas camadas demonstrado em Agung \textit{et al.} \cite{agung2023}, 
pode melhorar o desempenho do modelo uma vez que eles alcançaram uma precisão de 0,95 para o \textit{MobileNetV2}, 
enquanto este trabalho alcançou uma precisão mais baixa de 0,78 com a mesma arquitetura.

O uso de outros instrumentos de medição, como pluviômetros instalados por toda a cidade, pode ser usado para corroborar as classificações resultantes do modelo gerado. 
No entanto, como o escopo da pesquisa foi o desenvolvimento de um modelo de classificação e algumas ruas não possuem sensores de nível de água, 
esse tipo de análise foi deixada de lado para uma abordagem inicial.
