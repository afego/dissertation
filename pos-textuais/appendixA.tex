\chapter{Tabela de quantidade de images de cada classe para cada câmera identificada pelo seu código}
\label{apendA}

% \begin{table}[h]
%     \centering
%     %\caption{Exemplo de Tabela CSV}
%     \csvautotabular{dados/appendixa.csv}
%     \label{tab:apendA}
% \end{table}

% \begin{figure}[htb]
%     \centerline{\includegraphics[width=1\linewidth]{images/imagespercameratable.png}}
%     %\caption{Quantidade de imagens de normalidade por câmera disponível.}
%     \label{tab:apendA}
% \end{figure}

\begin{longtable}{|c|c|c|}
    \hline
    \textbf{Câmera} & \textbf{Classe 0} & \textbf{Classe 1} \\  
    \hline
    \endfirsthead
    
    \hline
    \textbf{Câmera} & \textbf{Classe 0} & \textbf{Classe 1} \\  
    \hline
    \endhead
    
    \csvreader[head to column names]{dados/appendixa.csv}{}%
    {\csvcoli & \csvcolii & \csvcoliii \\}%
\end{longtable}

% Este apêndice apresenta informações complementares.
% Elemento opcional. 
% O(s) apêndice(s) são identificados por letras maiúsculas consecutivas, travessão e pelos respectivos títulos. 
% Excepcionalmente utilizam-se letras maiúsculas dobradas, na identificação, quando esgotadas as 23 letras do alfabeto (ABNT, 2005).


